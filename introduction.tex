\chapter{Introduction}

\paragraph{Motivation}

The widespread adoption of machine learning systems that have opaque
decision-making mechanisms has lead to a growing number of privacy and fairness
concerns.

Most notably, recommender systems based on matrix factorization
input only the information about the behavior of users with regard to items
being recommended (e.g. movies on a movie rating website, products in an
online store) and from that they infer certain properties of the items and
user preferences in terms of these properties\cite{facebook-cf}. The properties
being inferred are unknown both to the system and to its users. Sometimes,
developers can eyeball these properties and conclude that, for example, a movie
recommender system, among other properties, infers whether a movie is more
female-oriented or male-oriented and whether it's more serious or escapist.
Correspondingly, it also infers whether each user prefers more ``masculine'' or
``feminine'' and more ``serious'' or ``escapist'' movies\cite{koren2009matrix}.
Unfortunately, such interpretations cannot always be automatically provided.
Automatically provided explanations are often expressed in terms of user or
item similarity (i.e. ``we recommend you item X because you liked items Y and
Z'') without giving insight into the internal inferences being made (such as
``we recommend you item X because we think that X is masculine and you like
masculine things''). While the gender-based targeting may or may not be seen as
problematic in the context of movie recommendations, but it is very likely to be
deemed problematic in the context of job recommendations, given the previous
history of controversies surrounding gendered targeting of job ads
\cite{datta_tschantz_datta_2015}.

Likewise, artificial neural networks can be used in credit decisions
\cite{WEST20001131}, predictive policing\cite{Seo2018PartiallyGN}, and other
applications where it is crucial (and often legally mandatory) to ensure
fairness. In practice, however, these networks can infer and use in their
decisions sensitive attributes such as gender even when they are not present
as features in the training data set\cite{Beutel2017DataDA}.

The issue is further complicated by the fact that the mere absence of usage of
sensitive protected attributes is not always sufficient to maintain legality:
if a certain selection criterion creates disparate impact on a sensitive
attribute (i.e. members of a protected group are being disproportionately
affected by it), it can be deemed illegal unless it can be justified as business
necessity. Furthermore, this can be specific to a particular predicate on a
feature rather than the feature itself: for example, it can be legal for an
airline to require candidates for pilot jobs to be at least 5'2" tall because
that is necessary to reach all of the aircraft controls but it is illegal to
set the minimum requirement at a higher point of 5'6" because that
disproportionately selects against women and is not justified as necessary
for a pilot\cite{RoseMaryCourtCase}.

\paragraph{Completed work}


