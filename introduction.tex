\chapter{Introduction}

\paragraph{Motivation}

The widespread adoption of machine learning systems that have opaque
decision-making mechanisms has lead to a growing number of privacy and fairness
concerns.

Most notably, recommender systems based on matrix factorization
input only the information about the behavior of users with regard to items
being recommended (e.g. movies on a movie rating website, products in an
online store) and from that they infer certain properties of the items and
user preferences in terms of these properties\cite{facebook-cf}. The properties
being inferred are unknown both to the system and to its users. Sometimes,
developers can eyeball these properties and conclude that, for example, a movie
recommender system, among other properties, infers whether a movie is more
female-oriented or male-oriented and whether it's more serious or escapist.
Correspondingly, it also infers whether each user prefers more ``masculine'' or
``feminine'' and more ``serious'' or ``escapist'' movies\cite{koren2009matrix}.
Unfortunately, such interpretations cannot always be automatically provided.
Automatically provided explanations are often expressed in terms of user or
item similarity (i.e. ``we recommend you item X because you liked items Y and
Z'') without giving insight into the internal inferences being made (such as
``we recommend you item X because we think that X is masculine and you like
masculine things''). While the gender-based targeting may or may not be seen as
problematic in the context of movie recommendations, but it is very likely to be
deemed problematic in the context of job recommendations, given the previous
history of controversies surrounding gendered targeting of job ads
\cite{datta_tschantz_datta_2015}.

Likewise, artificial neural networks can be used in credit decisions
\cite{WEST20001131}, predictive policing\cite{Seo2018PartiallyGN}, and other
applications where it is crucial (and often legally mandatory) to ensure
fairness. In practice, however, these networks can infer and use in their
decisions sensitive attributes such as gender even when they are not present
as features in the training data set\cite{Beutel2017DataDA}.

The issue is further complicated by the fact that the mere absence of usage of
sensitive protected attributes is not always sufficient to maintain legality:
if a certain selection criterion creates disparate impact on a sensitive
attribute (i.e. members of a protected group are being disproportionately
affected by it), it can be deemed illegal unless it can be justified as business
necessity. Furthermore, this can be specific to a particular predicate on a
feature rather than the feature itself: for example, it can be legal for an
airline to require candidates for pilot jobs to be at least 5'2" tall because
that is necessary to reach all of the aircraft controls but it is illegal to
set the minimum requirement at a higher point of 5'6" because that
disproportionately selects against women and is not justified as necessary
for a pilot\cite{RoseMaryCourtCase}.

\paragraph{Completed work}

In our study of interpreting latent factors of recommender systems, we introduce
a notion of a shadow model -- an interpretable model that mimics the behavior
of the original uninterpretable model as closely as possible. We are trying to
preserve both the external behavior and the internal structure of the original
model, and we show that these goals are not always equally achievable at the
same time. Specifically, we work with a MovieLens data set to create a
movie recommendation system based on matrix factorization. We then collect
metadata about movies present in the dataset from publicly available sources
such as IMDB and TV Tropes. We use that metadata to predict the values of latent
factors representing the properties of the movies. Then, we analyze these
predictions with quantitative input influence (QII) method
\cite{datta2016algorithmic} and use this analysis to infer the meaning of the
factors in the original model. This inference is further verified in experiments
with simulated users, for whom the true movie preferences are known, and thus
we can compare how accurately each explanation corresponds to properties the
users actually care about.

\paragraph{Proposed work}

A substantial limitation of the method outlined above is that not all
item properties inferred by the matrix factorization system can be interpreted.
This can be sometimes a limitation of the available metadata: for example,
if the data set contains tags about whether movies are oriented towards male or
female audience, the shadow model would be able to capture that meaning of a
latent factor, but if there are no tags about the movies being ``serious'' or
``escapist'', the shadow model would do its best at predicting its values from
other available data, which may result in overfitting or completely
unintelligible explanations. In practice, we have observed that most latent
factors tend to be predicted with roughly the same accuracy, which means that
all of them contain a mix of interpretable and uninterpretable information.




\paragraph{Thesis statement}

\emph{Representation aware accounting for matrix factorization models enables
explanations for for recommendations and mitigates bias.}

