\chapter{Isolating uninterpretability}

A major drawback of the method of explaining matrix factorization in terms of
known metadata features is that not all information encoded in the latent
factors is in fact interpretable. In our experiments, we have observed some
amount of variation in the quality of prediction of different latent factors,
but it is nowhere near as much as it would need to be in order for us to be able
to claim that some of these features are interpreted well, and others are
interpreted poorly. The matrix factorization model comes up with a distributed
representation of interpretable information, where all features together may
have information about several metadata factors, but every single feature has it
mixed together with uninterpretable information. Currently, the prediction
algorithms just to their best at predicting the latent factors and come up with
some possible interpretation of them, even if it is inaccurate.

Much better understanding of the model's behavior could be achieved if the
process of analysis specifically marked the parts that cannot be explained well
and gave more credible and confident explanations to the parts that can be
explained. We hypothesize that it can be achieved by transforming the model such
that the output remains unchanged (and thus we are still solving the task of
interpreting the existing model rather than creating a separate more
interpretable model), but the latent factors are altered in a way that makes
some of them much more predictable from metadata at the expense of the
predictability of the others. Then, we well-predicted factors can be
incorporated in the explanation, whereas the poorly predicted ones are discarded
and labeled as ``unknown''.

The idea that such transformation is possible is based on the assumption that
the information predictable from metadata is spread across multiple latent
factors and can be concentrated. For example, consider a situation when movies
are being recommended based on ``masculinity -- femininity'' axis and
``deep -- escapist'' axis, but the metadata only contains informations about
``masculinity -- femininity''. However, an infinite number of equivalent models
can be constructed by rotating the coordinates of both user and movie latent
factor matrices. Only four of these infinite possibilities will contain a latent
factor that is exactly aligned with the ``masculinity -- femininity'' axis,
which is therefore ideally predictable from the metadata. In all other possible
rotations, both of the latent factors will contain some information about both
``masculinity -- femininity'' and ``deep -- escapist'' axes. Assuming that
metadata doesn't contain information that could be used as a proxy for the
latter, the both latent factors will be predicted not ideally, involving some
information about ``masculinity -- femininity'' and then either nothing or
other features as the result of overfitting. The quality of explanation can be
drastically improved by finding such rotation of the coordinates that one of the
latent factors aligns with an axis predictable from metadata. The search for
such transformation can be formalized as trying to minimize the minimum
prediction error of the two latent factors.

In a more general case, the definitions of optimization criteria can be trickier
to develop. It is possible that no invertible transformation exists such that
one or more of the latent factors are predicted ideally. Instead, we would
expect to see variance in how well the factors are predicted. In addition, using
prediction quality as the optimization criterion can be challenging if the
model used for prediction is not gradient-based (e.g. a decision tree), and it
can be very computationally expensive to retrain models with every iteration of
the search. One possibility would be to optimize based on mutual information
between the latent factors and metadata features, but this is not a
straightforward task either due to the challenges in distribution estimation and
the fact that the space of metadata features is very highly dimensional.
Furthermore, even among models that predict a latent feature from metadata
ideally, simpler models are more desirable, being more human-understandable.
This can affect the transformation as well: if, for example, movies are
recommended based on whether they are horror movies, their year of release, and
whether they contain any footage of a burning fire, and the metadata set only
contains information about the first two, among transformations that result in
two of the latent factors being ideally predictable, there still exist many
possible rotations, some of which result in factors that are mix of horror and
release year, and some of which have them on orthogonal axes. The predictions
will be much more human-legible if the second case, but in order to find this
transformation, the simplicity of the model should also be incorporated into the
loss function, and there should be a trade-off between the model simplicity and
prediction quality.

Addressing all this challenges and coming up with a viable method for searching
for such transformations is the core of the proposed work.
